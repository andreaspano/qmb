
%%%%%%%%%%%%%%%%%%%%%%%%%%%%%%%%%%%%%%%%%%%%%%%%%%%%%%%%%%%%%%%%%%%%%%%%%%%%%%
%%%%%%%%%%%%%%%%%%%%%%%%%%%% VARIABILI DA DEFINIRE %%%%%%%%%%%%%%%%%%%%%%%%%%%
%%%%%%%%%%%%%%%%%%%%%%%%%%%%%%%%%%%%%%%%%%%%%%%%%%%%%%%%%%%%%%%%%%%%%%%%%%%%%%

% Titolo che appare nella barra in basso di ogni slide, al centro
% Se impostato ha la precedenza rispetto a quello di ogni singola slide

%\renewcommand{\titolocompleto}{}

\newcommand{\sottotitolo}{} % finche' non serve viene definito solo nel preambolo
% \newcommand{\data}{}



%%%%%%%%%%%%%%%%%%%%%%%%%%%%%%%%%%%%%%%%%%%%%%%%%%%%%%%%%%%%%%%%%%%%%%%%%%%%%%
%%%%%%%%%%%%%%%%%%%%%%%%%%%%%%%%%% PACKAGES %%%%%%%%%%%%%%%%%%%%%%%%%%%%%%%%%%
%%%%%%%%%%%%%%%%%%%%%%%%%%%%%%%%%%%%%%%%%%%%%%%%%%%%%%%%%%%%%%%%%%%%%%%%%%%%%%
\usepackage[latin1]{inputenc}   
\usepackage{graphicx}
\usepackage{rotating}
\usepackage{rotfloat}
\usepackage{color}
\usepackage{colortbl}
\usepackage{../../includeTex/floatflt}
\usepackage{tikz}
\usepackage{hyperref}
\usepackage{pgfpages} 
\usepackage{ifthen}



%%%%%%%%%%%%%%%%%%%%%%%%%%%%%%%%%%%%%%%%%%%%%%%%%%%%%%%%%%%%%%%%%%%%%%%%%%%%%%
%%%%%%%%%%%%%%%%%%%%%%%%%%%% IMPOSTAZIONI GENERALI %%%%%%%%%%%%%%%%%%%%%%%%%%%
%%%%%%%%%%%%%%%%%%%%%%%%%%%%%%%%%%%%%%%%%%%%%%%%%%%%%%%%%%%%%%%%%%%%%%%%%%%%%%

% Beamer theme
\usetheme{CambridgeUS}      

% Immagini da visualizzare
\newcommand{\materiale}{jmp}

% Path delle immagini
\graphicspath{{../../images/}}    

% Per caricare le formule matematiche con il giusto font 
% Questo sostituisce l'opzione mathserif di documentclass (obsoleta) 
\usefonttheme[onlymath]{serif}      



%%%%%%%%%%%%%%%%%%%%%%%%%%%%%%%%%%%%%%%%%%%%%%%%%%%%%%%%%%%%%%%%%%%%%%%%%%%%%%
%%%%%%%%%%%%%%%%%%%%%%%%%%%%%%%%%%% COLORI %%%%%%%%%%%%%%%%%%%%%%%%%%%%%%%%%%%
%%%%%%%%%%%%%%%%%%%%%%%%%%%%%%%%%%%%%%%%%%%%%%%%%%%%%%%%%%%%%%%%%%%%%%%%%%%%%%

\definecolor{grigio}{rgb}{0.46,0.48,0.48}
\definecolor{giallo}{rgb}{1,0.84,0}
\definecolor{coolred}{rgb}{0.83,0.06,0.27}
\definecolor{arancio}{rgb}{0.97,0.46,0.09}
\definecolor{verde}{rgb}{0.25,0.78,0.25}
\definecolor{qblu}{rgb}{0.24,0.27,0.74}
\definecolor{azzurro}{rgb}{0.37,0.91,0.90}

\definecolor{grigio}{rgb}{0.46,0.48,0.48}
\definecolor{blu}{rgb}{0.25,0.28,0.78}

\definecolor{sfondoScopo}{rgb}{0.75,0.785,0.83}
\definecolor{darkred}{named}{qblu}
\definecolor{blue}{named}{qblu}

\setbeamercolor{scopo}{bg=sfondoScopo}
\setbeamercolor{section in toc}{fg=black,bg=white}
\setbeamercolor{alerted text}{fg=darkred!80!gray}
\setbeamercolor{palette primary}{fg=darkred!60!black,bg=gray!30!white}
\setbeamercolor{palette secondary}{fg=darkred!70!black,bg=gray!15!white}
\setbeamercolor{palette tertiary}{bg=darkred!80!black,fg=gray!10!white}
\setbeamercolor{palette quaternary}{fg=darkred,bg=gray!5!white}

\setbeamercolor{sidebar}{fg=darkred,bg=gray!15!white}
\setbeamercolor{palette sidebar primary}{fg=darkred!10!black}
\setbeamercolor{palette sidebar secondary}{fg=white}
\setbeamercolor{palette sidebar tertiary}{fg=darkred!50!black}
\setbeamercolor{palette sidebar quaternary}{fg=gray!10!white}

\setbeamercolor{titlelike}{parent=pallette primary,fg=darkred}
\setbeamercolor{frametitle}{bg=gray!10!white}
\setbeamercolor{frametitle right}{bg=gray!60!white}

\setbeamercolor{separation line}{}
\setbeamercolor{fine separation line}{}

%% Definizione dei colori da assegnare ai box
\setbeamercolor{postit}{fg=white,bg=qblu}
\setbeamercolor{postut}{fg=qblu,bg=gray!60!white}

%% Definizione dei colori per i diagrammi
\definecolor{bloccoIniziale}{rgb}{0.94,0.93,0.48}
\definecolor{bloccoFinale}{rgb}{0.86,0.25,0.27}
\definecolor{blocco}{rgb}{0.56,0.58,0.77}
\definecolor{bloccoSospeso}{rgb}{0.94,0.81,0.36}



%%%%%%%%%%%%%%%%%%%%%%%%%%%%%%%%%%%%%%%%%%%%%%%%%%%%%%%%%%%%%%%%%%%%%%%%%%%%%%
%%%%%%%%%%%%%%%%%%%%%%%%% STRUTTURA DELLE DIAPOSITIVE %%%%%%%%%%%%%%%%%%%%%%%%
%%%%%%%%%%%%%%%%%%%%%%%%%%%%%%%%%%%%%%%%%%%%%%%%%%%%%%%%%%%%%%%%%%%%%%%%%%%%%%

% Intestazione
\setbeamertemplate{headline}
{
  \leavevmode%
  \hbox{%
  \begin{beamercolorbox}[wd=.5\paperwidth,ht=2.25ex,dp=1ex,right]{section in head/foot}%
    \usebeamerfont{section in head/foot}\insertsubsectionhead\hspace*{2ex}
  \end{beamercolorbox}%
  \begin{beamercolorbox}[wd=.5\paperwidth,ht=2.25ex,dp=1ex,left]{subsection in head/foot}%
    \usebeamerfont{subsection in head/foot}\hspace*{2ex}\insertsubsubsectionhead
  \end{beamercolorbox}}%
  \vskip0pt%
}

% Pie' di pagina
\setbeamertemplate{footline}
{
  \hbox{%
    \begin{beamercolorbox}[wd=.20\paperwidth, ht = 2.25ex, dp = 1ex, center]{palette sidebar secondary}%
      \usebeamerfont{author in head/foot}%\insertshortauthor~~(\insertshortinstitute) 
      \includegraphics[width=1.5cm]{QUANTIDE.png}
    \end{beamercolorbox}%
    \begin{beamercolorbox}[wd=.57\paperwidth, ht = 2.25ex, dp = 1ex, center]{title in head/foot}%
      \usebeamerfont{title in head/foot}{\titolocompleto \titoloshort}
    \end{beamercolorbox}%
    \begin{beamercolorbox}[wd=.13\paperwidth, ht = 2.25ex, dp = 1ex, left]{date in head/foot}%
      \hspace*{0.4em} \usebeamerfont{date in head/foot} {\numerocapitolo}
    \end{beamercolorbox}%
    \begin{beamercolorbox}[wd=.10\paperwidth, ht = 2.25ex, dp = 1ex, right]{date in head/foot}%
       \usebeamerfont{date in head/foot} \insertframenumber{} / \inserttotalframenumber \hspace*{2ex} 
    \end{beamercolorbox}%
  }%
  \vskip0pt%
}



%%%%%%%%%%%%%%%%%%%%%%%%%%%%%%%%%%%%%%%%%%%%%%%%%%%%%%%%%%%%%%%%%%%%%%%%%%%%%%
%%%%%%%%%%%%%%%%%%%%%%%%%%% STILE DELLE DIAPOSITIVE %%%%%%%%%%%%%%%%%%%%%%%%%%
%%%%%%%%%%%%%%%%%%%%%%%%%%%%%%%%%%%%%%%%%%%%%%%%%%%%%%%%%%%%%%%%%%%%%%%%%%%%%%

% Simboli di navigazione
\setbeamertemplate{navigation symbols}{}

% Modifica lo stile dell'elenco (di primo livello)
\useitemizeitemtemplate{$\star$} % Usa la stella

% Interlinea (fattore di scala; NON e' un valore assoluto)
\renewcommand{\baselinestretch}{1.2}  

% Definisci stile per vettori e matrici
\newcommand{\vect}[1]{\boldsymbol{\underline{#1}}} % Grassetto e sottolineato
\newcommand{\matr}[1]{\boldsymbol{#1}} % Grassetto

% Definire stile per valore assoluto
\providecommand{\abs}[1]{\lvert#1\rvert}
\providecommand{\norm}[1]{\lVert#1\rVert}

% Cambiare il nome delle figure e delle tabelle
\renewcommand{\figurename}{Figura}
\renewcommand{\tablename}{Tabella}

% Definizione sezioni, ecc.
\newcommand{\livelloA}{\subsection}
\newcommand{\livelloB}{\subsubsection}
\newcommand{\livelloC}{}

% Livello di profondita' del 'content panel' del PDF
\hypersetup{bookmarksdepth=4}

% Definizione prima slide
\title{\numerocapitolo\newline\textbf{\titolo}}
% \author{\sottotitolo}
\date{}


