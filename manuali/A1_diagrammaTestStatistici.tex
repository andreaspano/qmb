% Impostazioni principali
\documentclass[t, compress, mathserif]{beamer}



%         ----------------------------------------------         %
%        /                                              \        %
%--------               START PREAMBLE                   --------%
%        \                                              /        %
%         ----------------------------------------------         %

% Titolo che appare nella prima slide del documento
\newcommand{\titolo}{Decision Tree of statistical tests}
\newcommand{\sottotitolo}{Choice of the method according to the purpose of the data} 

% Titolo che appare nella barra in basso di ogni slide, al centro
% Sono due variabili:
% * una puo' essere utilizzata per l'intero corso. Se impostata nel preambolo sovrascrive quella di seguito.
% * l'altra puo' identificare ciascun documento
\newcommand{\titolocompleto}{Statistics course - }
\newcommand{\titoloshort}{\titolo}

% Numero di capitolo o altro nome che appare in basso di ogni slide, vicino al numero di pagina
\newcommand{\numerocapitolo}{Appendix 1}

% Include il documento che contiene il preambolo
\input{../includeTex/preamble.tex}



%         ----------------------------------------------         %
%        /                                              \        %
%--------               START DOCUMENT                   --------%
%        \                                              /        %
%         ----------------------------------------------         %

\begin{document}

% Pagina del titolo
\frame{\titlepage}

% Indice
% \begin{frame}
% 	 \tableofcontents
% \end{frame}

% Documento
% I soli contenuti del documento sono in un file esterno. Questo semplifica enormemente le cose qualora si volessero creare dei manuali (singoli documenti) a partire da diversi documenti.
\livelloA{Chart 1}

\begin{frame}

\onslide<1>{
\begin{tiny}
	\begin{tikzpicture}[level distance=10mm]
		
                \tikzstyle{every node}=[fill=blocco!50,rounded corners,text width=1.2cm]
                \tikzstyle{edge from parent}=[blocco,thick,draw]
		\tikzstyle{level 1}=[sibling distance=28mm]
  		\tikzstyle{level 2}=[sibling distance=24mm]
  		\tikzstyle{level 3}=[sibling distance=42mm]
  		\tikzstyle{level 4}=[sibling distance=30mm]
		\tikzstyle{level 5}=[level distance=14mm]
		\tikzstyle{level 6}=[level distance=10mm]
                  \node[text width=3.4cm] [fill=blocco]{\textbf{How many variables are analyzed?}}
			child {node {\textbf{one}}
				child {node [text width=2.2cm] {Is It quantitative?}
					child {node [text width=2.2cm]{\textbf{yes}. \\ What do you want to test?}
						child {node[text width=2.2cm] {\textbf{Mean equal to a value}}
							child {node {Sigma known}
								child {node[fill=bloccoFinale!50] {\textbf{Test Z}}}}
							child {node {Sigma NOT known}
								child {node [fill=bloccoFinale!50]{\textbf{One sample T test}}}}}
						child {node [text width=2.2cm]{\textbf{Variance equal to a value}}
							child {node [fill=bloccoFinale!50]{\textbf{Test for one variance} $\chi^2$}}}}
					child {node [text width=2.2cm]{\textbf{No}.\\ What do you want to test?}
						child {node [text width=2.2cm] {\textbf{True percentage equal to a value}}
							child {node[text width=2.2cm] [fill=bloccoFinale!50]{\textbf{Tests on percentages (tests on one proportion, $\chi^2$ ``observed vs expected''}) }}}}}}
			child {node{\textbf{two}}child {node[ text width=1.3cm] [fill=bloccoSospeso!50]{See chart 2}}}
			child {node{ $>$\textbf{two}}child {node[ text width=1.3cm] [fill=bloccoSospeso!50]{See chart 4}}};

         \end{tikzpicture}
\end{tiny}
}

\end{frame}

\livelloA{Chart 2}
\begin{frame}

\onslide<1>{
\begin{tiny}
	\begin{tikzpicture}[level distance=10mm]
                \tikzstyle{every node}=[fill=blocco!50,rounded corners,text width=1.2cm]
                \tikzstyle{edge from parent}=[blocco,thick,draw]
		\tikzstyle{level 1}=[sibling distance=28mm]
  		\tikzstyle{level 3}=[sibling distance=50mm]
  		\tikzstyle{level 4}=[sibling distance=18mm]
  		\tikzstyle{level 5}=[level distance=15mm]
                  \node [ text width=3.4cm] [fill=blocco]{\textbf{How many variables are analyzed?}}
			child {node{\textbf{one}}child {node [ text width=1.3cm][fill=bloccoSospeso!50]{See chart 1}}}
			child {node {\textbf{two}}
				child {node  {What kind of data is used?}
					child {node {\textbf{both \\ qualitative}}
						child {node [ text width=2cm]{What is It to be tested?}
							child {node {\textbf{Agreement}}
								child {node[fill=bloccoFinale!50] {\textbf{K of Cohen} (or others)}}}
							child {node {\textbf{Association Addiction}}
								child {node [fill=bloccoFinale!50]{\textbf{$\chi^2$} Test\\(or similar)}}}}}
					child {node {\textbf{both \\ quantitative}}
						child {node [ text width=2cm]{What is It to be tested?}
							child {node {\textbf{Comparison of means (no repeated measures)}}
								child {node [fill=bloccoFinale!50]{\textbf{Two-samples t Test}}}}
							child {node {\textbf{Comparison of means \\(repeated measures)}}
								child {node [fill=bloccoFinale!50]{\textbf{Paired t Test}}}}
							child {node {\textbf{Association}}
								child {node [fill=bloccoFinale!50]{\textbf{Pearson or Spearman Correlation}}}}
							child {node {\textbf{Dependence}}
								child {node [fill=bloccoFinale!50]{\textbf{Simple \\ Regression or linear models}}}}}}
					child {node {\textbf{one \\ qualitative \\ and one quantitative}}child {node [ text width=1.3cm] [fill=bloccoSospeso!50]{See chart 3}}}}}
			child {node{$>$\textbf{two}}child {node[ text width=1.3cm] [fill=bloccoSospeso!50]{See chert 4}}};
         \end{tikzpicture}
\end{tiny}
}

\end{frame}


\livelloA{Chart 3}
\begin{frame}

\onslide<1>{
\begin{tiny}
	\begin{tikzpicture}[level distance=10mm]
                \tikzstyle{every node}=[fill=blocco!50,rounded corners,text width=1.2cm]
                \tikzstyle{edge from parent}=[blocco,thick,draw]
		\tikzstyle{level 1}=[sibling distance=40mm]
  		\tikzstyle{level 2}=[sibling distance=30mm]
  		\tikzstyle{level 3}=[sibling distance=60mm]
  		\tikzstyle{level 5}=[sibling distance=20mm]
  		\tikzstyle{level 6}=[sibling distance=70mm]
		\tikzstyle{level 6}=[level distance=10mm]
  		\tikzstyle{level 7}=[level distance=15mm]
                  \node[ text width=4.8cm][fill=bloccoSospeso!50] {\textbf{2 variables: one qualitative and one quantitative}}
			child {node [ text width=4.5cm] {Does a \textbf{dependent} variable exist?}
			child {node {\textbf{no}}
				child {node [fill=bloccoFinale!50]{\textbf{Association indexes}}}}
			child {node [ text width=2cm] {\textbf{yes}, the quantitative is the dependent}
				child {node [ text width=3cm]{What do you want to check?}
					child {node[ text width=3cm] {\textbf{Difference in variance of the dependent v. varying the values of the qualitative v.}}
						child {node [ text width=3cm] {Number of values assumed by qualitative variable}
						child {node {\textbf{2}}
							child {node[fill=bloccoFinale!50] {\textbf{Test F or Levene Test}}}}
						child {node {$>$\textbf{2}}
							child {node[fill=bloccoFinale!50] {\textbf{Bartlett o Levene Test}}}}
						child {node {\textbf{virtually unlimited}}
							child {node [fill=bloccoFinale!50]{\textbf{Breush-Pagan Test and others}}}}}}
					child {node [ text width=3cm] {\textbf{Difference in mean of the dependent v. varying the values of the qualitative v.}}
						child {node [ text width=3cm] {Number of values assumed by qualitative variable}
						child {node {\textbf{2}}
							child {node [ text width=2cm][fill=bloccoFinale!50]{\textbf{Student's t test for independent groups or one-way ANOVA with fixed effects for indipendent groups}}}}
						child {node {$>$\textbf{2}}
							child {node[fill=bloccoFinale!50] {\textbf{one-way ANOVA with fixed effects for indipendent groups}}}}
						child {node {\textbf{virtually unlimited}}
							child {node[fill=bloccoFinale!50][ text width=2cm] {\textbf{Regression, one-way ANOVA with Random effects for indipendent groups} }}}}
				}}}
			child {node [ text width=2cm]{\textbf{yes}, the qualitative is the dipendent}
				child {node[fill=bloccoFinale!50] [ text width=2cm] {\textbf{Logit, Probit \\ Analysis (Generalized linear models)}}}}};
         \end{tikzpicture}
\end{tiny}
}
\end{frame}

\livelloA{Chart 4}
\begin{frame}
\onslide<1>{
\begin{tiny}
	\begin{tikzpicture}[level distance=10mm]
                \tikzstyle{every node}=[fill=blocco!50,rounded corners,text width=1.2cm]
                \tikzstyle{edge from parent}=[blocco,thick,draw]
		\tikzstyle{level 2}=[sibling distance=70mm]
  		\tikzstyle{level 3}=[sibling distance=40mm]
  		\tikzstyle{level 4}=[level distance=15mm]
  		%\tikzstyle{level 3}=[sibling distance=60mm]
                  \node[ text width=6cm][fill=bloccoSospeso!50] {\textbf{$>$2 variables.} \\ It is hypothesized that one of the variables may depend from the other}
			child {node [ text width=4.5cm]{What type of data has the dependent variable?}
				child {node [ text width=2.5cm]{\textbf{Qualitative o discrete quantitative}}
					child {node [ text width=2.5cm] {What do you want to check?}
						child {node [ text width=2.5cm]{\textbf{Dependence of the mean of the dependent variable from other variables}}
							child {node [ text width=2.5cm] [fill=bloccoFinale!50]{\textbf{Generalized linear models or, in specific cases, multiple regression and generalized linear models}}}}}}
				child {node {\textbf{Continuous}}
					child {node [ text width=2.5cm]{What do you want to check?}
						child {node [ text width=2.5cm]{\textbf{Dependence of the variance of the dependent variable from other variables}}
							child {node [ text width=2.5cm][fill=bloccoFinale!50]{\textbf{Breush-Pagan Test or others}}}}
						child {node [ text width=2.5cm]{\textbf{Dependence of the mean of the dependent variable from other variables}}
							child {node [ text width=2.5cm][fill=bloccoFinale!50] {\textbf{Multiple regression,\\  general linear models, generalized linear models}}}}}}};
		\end{tikzpicture}
\end{tiny}
}
\end{frame}





\end{document}
